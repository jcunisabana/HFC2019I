\documentclass[spanish,]{article}
\usepackage{lmodern}
\usepackage{amssymb,amsmath}
\usepackage{ifxetex,ifluatex}
\usepackage{fixltx2e} % provides \textsubscript
\ifnum 0\ifxetex 1\fi\ifluatex 1\fi=0 % if pdftex
  \usepackage[T1]{fontenc}
  \usepackage[utf8]{inputenc}
\else % if luatex or xelatex
  \ifxetex
    \usepackage{mathspec}
  \else
    \usepackage{fontspec}
  \fi
  \defaultfontfeatures{Ligatures=TeX,Scale=MatchLowercase}
\fi
% use upquote if available, for straight quotes in verbatim environments
\IfFileExists{upquote.sty}{\usepackage{upquote}}{}
% use microtype if available
\IfFileExists{microtype.sty}{%
\usepackage{microtype}
\UseMicrotypeSet[protrusion]{basicmath} % disable protrusion for tt fonts
}{}
\usepackage[margin=1in]{geometry}
\usepackage{hyperref}
\hypersetup{unicode=true,
            pdftitle={Historia y filosofía de la ciencia},
            pdfborder={0 0 0},
            breaklinks=true}
\urlstyle{same}  % don't use monospace font for urls
\ifnum 0\ifxetex 1\fi\ifluatex 1\fi=0 % if pdftex
  \usepackage[shorthands=off,main=spanish]{babel}
\else
  \usepackage{polyglossia}
  \setmainlanguage[]{spanish}
\fi
\usepackage{graphicx,grffile}
\makeatletter
\def\maxwidth{\ifdim\Gin@nat@width>\linewidth\linewidth\else\Gin@nat@width\fi}
\def\maxheight{\ifdim\Gin@nat@height>\textheight\textheight\else\Gin@nat@height\fi}
\makeatother
% Scale images if necessary, so that they will not overflow the page
% margins by default, and it is still possible to overwrite the defaults
% using explicit options in \includegraphics[width, height, ...]{}
\setkeys{Gin}{width=\maxwidth,height=\maxheight,keepaspectratio}
\IfFileExists{parskip.sty}{%
\usepackage{parskip}
}{% else
\setlength{\parindent}{0pt}
\setlength{\parskip}{6pt plus 2pt minus 1pt}
}
\setlength{\emergencystretch}{3em}  % prevent overfull lines
\providecommand{\tightlist}{%
  \setlength{\itemsep}{0pt}\setlength{\parskip}{0pt}}
\setcounter{secnumdepth}{0}
% Redefines (sub)paragraphs to behave more like sections
\ifx\paragraph\undefined\else
\let\oldparagraph\paragraph
\renewcommand{\paragraph}[1]{\oldparagraph{#1}\mbox{}}
\fi
\ifx\subparagraph\undefined\else
\let\oldsubparagraph\subparagraph
\renewcommand{\subparagraph}[1]{\oldsubparagraph{#1}\mbox{}}
\fi

%%% Use protect on footnotes to avoid problems with footnotes in titles
\let\rmarkdownfootnote\footnote%
\def\footnote{\protect\rmarkdownfootnote}

%%% Change title format to be more compact
\usepackage{titling}

% Create subtitle command for use in maketitle
\newcommand{\subtitle}[1]{
  \posttitle{
    \begin{center}\large#1\end{center}
    }
}

\setlength{\droptitle}{-2em}

  \title{Historia y filosofía de la ciencia}
    \pretitle{\vspace{\droptitle}\centering\huge}
  \posttitle{\par}
    \author{}
    \preauthor{}\postauthor{}
    \date{}
    \predate{}\postdate{}
  
\usepackage{booktabs}
\usepackage{longtable}
\usepackage{array}
\usepackage{multirow}
\usepackage{wrapfig}
\usepackage{float}
\usepackage{colortbl}
\usepackage{pdflscape}
\usepackage{tabu}
\usepackage{threeparttable}
\usepackage{threeparttablex}
\usepackage[normalem]{ulem}
\usepackage{makecell}
\usepackage{xcolor}

\usepackage{fontspec}
\setmainfont{Adobe Jenson Pro}
\linespread{1.05}

\begin{document}
\maketitle

\subsection{Descripción del curso}\label{descripcion-del-curso}

La filosofía de la ciencia trata los problemas fundamentales que surgen
al interior de la ciencia y la práctica ciencitífica. Puede ser dividia
en dos grandes ramas: la filosofía de la ciencia general y las filosofía
de ciencias específicas.

La primera busca entender la ciencia como una actividad que está en la
posición privilegiada de ofrecer conocimiento del mundo; la segunda
busca entender preguntas particulares que nacen dentro de las diferentes
ramas de la ciencia.

El objetivo central de este curso es familiarizar al estudiante con
preguntas fundamentales en filosofía de la ciencia general y algunas de
las problemáticas que surgen dentro de las ciencias particulares y de
las que se ocupan las filosofías de las ciencias específicas.

\textbf{Profesor}: \href{../index.html}{Juan Camilo Espejo-Serna}~

\textbf{Horario y salón}: \textbf{Miércoles}, 8:00 - 9:00 am, E203; \textbf{Viernes} 8:00
- 10:00 am, C106.

\textbf{Página web del seminario}:
\url{https://jcunisabana.github.io/HFC2019I/}

\begin{center}\rule{0.5\linewidth}{\linethickness}\end{center}

\subsection{Objetivos}\label{objetivos}

\begin{itemize}
\item
  Comprender el lenguaje propio de la filosofía para aportar en las
  discusiones sobre la ciencia
\item
  Distinguir, relacionar y sistematizar conocimientos aportados por la
  ciencia específicas y la filosofía, para dar cuenta epistemológica de
  las ciencia en general y superar la fragmentación.
\item
  Planear y elaborar textos interpretativos y argumentativos con base en
  historia y la filosofía de la ciencia
\item
  Utilizar TIC para apoyar el estudio filosófico de la ciencia.
\end{itemize}

\begin{center}\rule{0.5\linewidth}{\linethickness}\end{center}

\subsection{Metodología}\label{metodologia}

\paragraph{\texorpdfstring{\textbf{Antes de la
sesión}}{Antes de la sesión}}\label{antes-de-la-sesion}

\begin{itemize}
\tightlist
\item
  Todos los estudiantes deberán subir un control de lectura por tarde
  \textbf{75 horas} antes de la sesión.
\end{itemize}

\paragraph{\texorpdfstring{\textbf{Durante la
sesión}}{Durante la sesión}}\label{durante-la-sesion}

\begin{itemize}
\tightlist
\item
  Todos deben atender con cuidado a la presentación del profesor y
  formular preguntas al respecto. Revisen si entienden la exposición y
  si están de acuerdo; pregunten por las relaciones con los temas
  anteriormente expuestos.
\end{itemize}

\begin{center}\rule{0.5\linewidth}{\linethickness}\end{center}

\subsection{Plan semanal}\label{plan-semanal}

\subsubsection{Semana 1}\label{semana-1}

\begin{tabular}{| >{\raggedright\arraybackslash}p{10em}|>{\raggedright\arraybackslash}p{10em}|} 
\hline
\textbf{\textbf{Miércoles}} & \textbf{\textbf{Viernes}}\\
\hline
Presentación del programa & Las preguntas de la filosofía de la ciencia\\
\hline
\end{tabular}

Presentación en pantalla completa

\begin{center}\rule{0.5\linewidth}{\linethickness}\end{center}

\subsubsection{Semana 2}\label{semana-2}

\begin{tabular}{| >{\raggedright\arraybackslash}p{10em}|>{\raggedright\arraybackslash}p{10em}|} \hline
\textbf{Miércoles} & \textbf{Viernes}\\
\hline
La explicación según el aristotelismo & Los inicios del método axiomático\\
\hline
\end{tabular}

\begin{itemize}
\item
  Leer: Losee, J. (1976). Introducción histórica a la filosofía de la
  ciencia. España: Alianza Editorial. Pags. 15-38
\item
  Hacer: Control de lectura
\end{itemize}

\begin{center}\rule{0.5\linewidth}{\linethickness}\end{center}

\subsubsection{Semana 3}\label{semana-3}

\begin{tabular}{|>{\raggedright\arraybackslash}p{10em}|>{\raggedright\arraybackslash}p{10em}|}\hline
\textbf{Miércoles} & \textbf{Viernes}\\
\hline
El universo según el aristotelismo & Críticas al aristotelismo\\
\hline
\end{tabular}

\begin{itemize}
\item
  Leer: Losee, J. (1976). Introducción histórica a la filosofía de la
  ciencia. España: Alianza Editorial. Pags. 53-103
\item
  Hacer: Control de lectura
\end{itemize}

\begin{center}\rule{0.5\linewidth}{\linethickness}\end{center}

\subsubsection{Semana 4}\label{semana-4}

\begin{tabular}{|>{\raggedright\arraybackslash}p{10em}|>{\raggedright\arraybackslash}p{10em}|}
\hline
\textbf{Miércoles} & \textbf{Viernes}\\
\hline
Física clásica & Físicas no-clásicas: relativista y cuántica\\
\hline
\end{tabular}

\begin{itemize}
\item
  Leer: Fine, A. (1986). The Shaky Game. Chicago, USA: The
  University of Chicago Press. Caps 1, 3 y 5.
\item
  Hacer: Control de lectura y el taller en virtual sabana
\end{itemize}

\begin{center}\rule{0.5\linewidth}{\linethickness}\end{center}

\subsubsection{Semana 5}\label{semana-5}

\begin{tabular}{|>{\raggedright\arraybackslash}p{10em}|>{\raggedright\arraybackslash}p{10em}|}
\hline
\textbf{Miércoles} & \textbf{Viernes}\\
\hline
Deducción e inducción & Problemas para la deducción y la inducción\\
\hline
\end{tabular}

\begin{itemize}
\item
  Leer: Moulines, C. U. (1993). La ciencia: estructura y desarrollo.
  Madrid, España: Trotta. Cap 2.
\item
  Hacer: Control de lectura
\end{itemize}

\begin{center}\rule{0.5\linewidth}{\linethickness}\end{center}

\subsubsection{Semana 6}\label{semana-6}

\begin{tabular}{|>{\raggedright\arraybackslash}p{10em}|>{\raggedright\arraybackslash}p{10em}|}
\hline
\textbf{Miércoles} & \textbf{Viernes}\\
\hline
La concepción heredada de las teorías científicas I & La concepción heredada de las teorías científicas I\\
\hline
\end{tabular}

\begin{itemize}
\item
  Leer: Suppe, Frederick (1979)~La estructura de las teorías
  científicas. Editora Nacional: Madrid, España. Partes I, II (★) y III
\item
  Hacer: Control de lectura
\end{itemize}

\begin{center}\rule{0.5\linewidth}{\linethickness}\end{center}

\subsubsection{Semana 7}\label{semana-7}

\begin{tabular}{|>{\raggedright\arraybackslash}p{10em}|>{\raggedright\arraybackslash}p{10em}|}
\hline
\textbf{Miércoles} & \textbf{Viernes}\\
\hline
La concepción heredada de las teorías científicas II & La concepción heredada de las teorías científicas II\\
\hline
\end{tabular}

\begin{itemize}
\item
  Leer: Suppe, Frederick (1979) La estructura de las teorías
  científicas. Editora Nacional: Madrid, España. Partes I, II y III
\item
  Hacer: Control de lectura
\end{itemize}

\begin{center}\rule{0.5\linewidth}{\linethickness}\end{center}

\subsubsection{Semana 8}\label{semana-8}

\begin{tabular}{|>{\raggedright\arraybackslash}p{10em}|>{\raggedright\arraybackslash}p{10em}|}
\hline
\textbf{Miércoles} & \textbf{Viernes}\\
\hline
Problemas de la concepción heredada & Quine: Dos dogmas del empirismo\\
\hline
\end{tabular}

\begin{itemize}
\item
  Leer: Quine, W. V. O. (2002) Desde un punto de vista lógico. Paidos:
  Barcelona, España. Capítulo 2.
\item
  Hacer: Control de lectura
\end{itemize}

\begin{center}\rule{0.5\linewidth}{\linethickness}\end{center}

\subsubsection{Semana 9}\label{semana-9}

\begin{tabular}{|>{\raggedright\arraybackslash}p{10em}|>{\raggedright\arraybackslash}p{10em}|}
\hline
\textbf{Miércoles} & \textbf{Viernes}\\
\hline
Popper: contra la concepción heredada & Popper: conjeturas y refutaciones\\
\hline
\end{tabular}

\begin{itemize}
\item
  Leer: Popper, K (2002) Conjeturas y refutaciones. Paidos: Barcelona,
  España. Capítulo 1.
\item
  Hacer: Control de lectura
\end{itemize}

\begin{center}\rule{0.5\linewidth}{\linethickness}\end{center}

\subsubsection{Semana 10}\label{semana-10}

\begin{tabular}{|>{\raggedright\arraybackslash}p{10em}|>{\raggedright\arraybackslash}p{10em}|}
\hline
\textbf{Miércoles} & \textbf{Viernes}\\
\hline
Paradigmas científicos I & Paradigmas científicos II\\
\hline
\end{tabular}

\begin{itemize}
\item
  Leer: Kuhn, Thomas (1962)~La estructura de las revoluciones
  científicas. Fondo de cultura económica: México. Capítulos 1, 2, 3, 4, 5 y 6 
\item
  Hacer: Control de lectura y el taller en virtual sabana
\end{itemize}

\begin{center}\rule{0.5\linewidth}{\linethickness}\end{center}

\subsubsection{Semana 11}\label{semana-11}

\begin{tabular}{|>{\raggedright\arraybackslash}p{10em}|>{\raggedright\arraybackslash}p{10em}|}
\hline
\textbf{Miércoles} & \textbf{Viernes}\\
\hline
Revoluciones científicas I & Revoluciones científicas II\\
\hline
\end{tabular}

\begin{itemize}
\item
  Leer: Kuhn, Thomas (1962)~La estructura de las revoluciones
  científicas. Fondo de cultura económica: México. Capítulos 6, 7,
  8, 9 y 10.
\item
  Hacer: Control de lectura
\end{itemize}

\begin{center}\rule{0.5\linewidth}{\linethickness}\end{center}

\subsubsection{Semana 12}\label{semana-12}

\begin{tabular}{|>{\raggedright\arraybackslash}p{10em}|>{\raggedright\arraybackslash}p{10em}|}
\hline
\textbf{Miércoles} & \textbf{Viernes}\\
\hline
Ciencia y valores & Ciencia y valores\\
\hline
\end{tabular}

\begin{itemize}
\item
  Leer: (Por definir)
\item
  Hacer: Control de lectura
\end{itemize}

\begin{center}\rule{0.5\linewidth}{\linethickness}\end{center}

\subsubsection{Semana 13}\label{semana-13}

\begin{tabular}{|>{\raggedright\arraybackslash}p{10em}|>{\raggedright\arraybackslash}p{10em}|}
\hline
\textbf{Miércoles} & \textbf{Viernes}\\
\hline
Semana santa & NA\\
\hline
\end{tabular}

\begin{center}\rule{0.5\linewidth}{\linethickness}\end{center}

\subsubsection{Semana 14}\label{semana-14}

\begin{tabular}{|>{\raggedright\arraybackslash}p{10em}|>{\raggedright\arraybackslash}p{10em}|}
\hline
\textbf{Miércoles} & \textbf{Viernes}\\
\hline
Ciencia y valores & Ciencia y valores\\
\hline
\end{tabular}

\begin{itemize}
\item
  Leer: (Por definir)
\item
  Hacer: Control de lectura
\end{itemize}

\begin{center}\rule{0.5\linewidth}{\linethickness}\end{center}

\subsubsection{Semana 15}\label{semana-15}

\begin{tabular}{|>{\raggedright\arraybackslash}p{10em}|>{\raggedright\arraybackslash}p{10em}|}
\hline
\textbf{Miércoles} & \textbf{Viernes}\\
\hline
Filosofía de las ciencias específicas & Filosofía de las ciencias específicas\\
\hline
\end{tabular}

\begin{itemize}
\item
  Leer: (Por definir)
\item
  Hacer: Control de lectura
\end{itemize}

\begin{center}\rule{0.5\linewidth}{\linethickness}\end{center}

\subsubsection{Semana 16}\label{semana-16}

\begin{tabular}{|>{\raggedright\arraybackslash}p{10em}|>{\raggedright\arraybackslash}p{10em}|}
\hline
\textbf{Miércoles} & \textbf{Viernes}\\
\hline
Filosofía de las ciencias específicas & Filosofía de las ciencias específicas\\
\hline
\end{tabular}

\begin{itemize}
\item
  Leer: (Por definir)
\item
  Hacer: Control de lectura
\end{itemize}

\begin{center}\rule{0.5\linewidth}{\linethickness}\end{center}

\subsubsection{Semana 17}\label{semana-17}

\begin{tabular}{|>{\raggedright\arraybackslash}p{10em}|>{\raggedright\arraybackslash}p{10em}|}
\hline
\textbf{Miércoles} & \textbf{Viernes}\\
\hline
Filosofía de las ciencias específicas & Filosofía de las ciencias específicas\\
\hline
\end{tabular}

\begin{itemize}
\item
  Leer: (Por definir)
\item
  Hacer: Control de lectura
\end{itemize}

\begin{center}\rule{0.5\linewidth}{\linethickness}\end{center}

\subsubsection{Semana 18}\label{semana-18}

\begin{tabular}{|>{\raggedright\arraybackslash}p{10em}|>{\raggedright\arraybackslash}p{10em}|}
\hline
\textbf{Miércoles} & \textbf{Viernes}\\
\hline
Repaso & Nada\\
\hline
\end{tabular}

\begin{itemize}
\item
  Leer: ¡Todo!
\item
  Hacer: el examen final en virtual sabana
\end{itemize}

\begin{center}\rule{0.5\linewidth}{\linethickness}\end{center}

\subsection{Evaluación}\label{evaluacion}

\paragraph{\texorpdfstring{\textbf{Talleres}}{Talleres}}\label{talleres}

Los talleres consistirán en una serie de preguntas que los alumnos
deberán solucionar en la plataforma virtual. Es deber del estudiante
entender bien cómo funciona la plataforma con anticipación a la fecha
límite de entrega del taller.

\paragraph{\texorpdfstring{\textbf{Control de
lectura}}{Control de lectura}}\label{control-de-lectura}

Extensión: entre 400 y 1000 palabras.

Para cada lectura asignada, los estudiantes deben escribir un texto
corto con la tesis principal, tres afirmaciones/presuposiciones del
texto y tres preguntas/desafíos al texto.

Los controles deberán ser subidos a la plataforma virtual a más tardar
\textbf{75 horas} antes de la sesión. Todos los estudiantes empiezan con
5.0 en esta nota; por cada vez que no se participe dentro del rango de
tiempo especificado, la nota será disminuida de acuerdo con los
siguientes parámetros: primera vez: -0.5; segunda vez: -1.0; tercera
vez: -1.5; cuarta vez: -2.0.

Todos tienen un control de lectura ``de gracia''. Es decir, pueden dejar
de entregar uno sin problema; el primer control de lectura que les falte
no cuenta. Por ejemplo, si no entregan un control de lectura y entregan
todos los demás, su nota igual queda en 5.0.

\paragraph{\texorpdfstring{\textbf{Calificación}}{Calificación}}\label{calificacion}


\begin{table}[h]
\begin{tabular}{|l|c|c|}
	\hline
	
\textbf{Actividad}            & \textbf{Porcentaje} & \textbf{Corte} \\
\hline
Taller               & 15\%       & 1     \\
Controles de lectura & 15\%       & 1     \\
Taller               & 15\%       & 2     \\
Controles de lectura & 15\%       & 2     \\
Controles de lectura & 15\%       & 3     \\
Examen               & 25\%       & 3    \\
\hline

\end{tabular}
\end{table}

\textbf{Toda} entrega tarde injustificada verá la nota disminuida en 0.5
unidades por cada día tarde. No haber entregado antes de la hora
acordada equivale a entregar un día tarde.


\end{document}
